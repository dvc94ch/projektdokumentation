\documentclass[a4paper]{report}

% Use swiss german letters
\usepackage[utf8]{inputenc}

% Language: german
\usepackage[ngerman]{babel}

% Fancy Figures
\usepackage{graphicx}

% Use Times
\usepackage{mathptmx}

% Display the Bibliography in the TOC
\usepackage{tocbibind}

% Better lists
\usepackage{enumitem}

% Use biblatex
\usepackage[style=apa,backend=biber,citestyle=authoryear]{biblatex} 

% Tell BibLatex to use the ngerman language mapping
\DeclareLanguageMapping{ngerman}{ngerman-apa}

% Define the bibliography file
\addbibresource{bibliography.bib}

% To let LaTeX handle "
\usepackage[autostyle=true, german=quotes]{csquotes}

% Titlepage
\newcommand*{\titleAP}{\begingroup % Create the command for including the title page in the document
	\centering
	\vspace*{\baselineskip} % Whitespace at the top of the page
	
	{\Large FirstName LastName}\\[0.167\textheight] % Author name
	
	{\Huge\bfseries Projektdokumentation PREN Gruppe03}\\[\baselineskip]
	
	%TODO review subtitle
	{\Large \textit{Subtitle}}\\
	\today
	
	\vspace*{3\baselineskip} % Whitespace at the bottom of the page
	\endgroup}

% Define the path were images are found
\graphicspath{{./img/}}

\begin{document}

\titleAP

\newpage

\begin{abstract}
	Hier würde man das Abstract oder Management Summary schreiben.
	Testibus Eve
\end{abstract}

\tableofcontents

\newpage

\chapter{Einleitung}
\label{ch:Intro}

\section{Präambel}
Eine Einleitung in ein \LaTeX Dokument. 
Dieses ist im Moment auf Github gehosted, Git ist ein Codeverwaltungsprogramm welches von vielen Programmierern und OpenSource Projekten benutzt wird. \parencite{Git2017}

Ein leichtverständliches Buch wurde von \citeauthor{Chacon2016} geschrieben und ist frei verfügbar.

In diesem Satz verweise ich auf Bild \ref{fig:MemeMagic}, welches mir \LaTeX automatisch richtig beschriftet.

\begin{figure}[h!]
	\centering
	\includegraphics[width=0.5\linewidth,keepaspectratio]{MemeMagic}
	\caption{\LaTeX is magic - sometimes black magic...}
	\label{fig:MemeMagic}
	
\end{figure}

\chapter{Projektorganisation}

\section{Teamübersicht}

\begin{tabular}{|l|l|}
	\hline 
	\textbf{Name} & \textbf{Studium} \\ 
	\hline 
	Pascal Baumann & Informatik \\ 
	\hline 
	Basil Bachmann & Maschinenbau \\ 
	\hline 
	David Craven & Elektrotechnik \\ 
	\hline 
	Victor Guntern & Maschinenbau \\ 
	\hline
	Markus Kempf & Maschinenbau \\ 
	\hline  
	Eve Meier & Informatik \\ 
	\hline 
	Jan Odermatt & Elektrotechnik \\
	\hline
	Simon Rohrer & Maschinenbau \\
	\hline
\end{tabular} 

\section{Projektrollen}

\begin{tabular}{|l|l|}
	\hline 
	\textbf{Rolle} & \textbf{Teammitglied} \\ 
	\hline 
	Projektleiter & Eve \\ 
	\hline 
	Projektplaner & Markus \\ 
	\hline 
	Verantwortlicher Dokumentation & Pascal \\ 
	\hline 
	Fachverantwortlicher I & Pascal \\ 
	\hline 
	Fachverantwortlicher ET & Jan \\ 
	\hline 
	Fachverantwortlicher MB & Markus \\ 
	\hline 
\end{tabular} 

\chapter{Anforderungen}
\section{Projektanforderungen}
\begin{tabular}{|l|l|l|}
	\hline 
	Nr. & Bezeichnung & Beschreibung \\ 
	\hline 
	1.1 & Projektabgabe & Dezember 2017 \\ 
	\hline 
	1.2 & Budget f. PREN & Max. 500CHF \\ 
	\hline 
	1.3 & Teilbudget PREN1 & Max. 200CHF \\ 
	\hline 
	1.4 & Eigenleistung &  \\ 
	\hline 
	1.5 & 3D-Drucker Laufzeit & max. 25 \\ 
	\hline 
	1.6 & Lasergerät Laufzeit & max. 1 \\ 
	\hline 
	1.7 & Stunden ET-Werkstattpersonal & max. 10 \\ 
	\hline 
	1.8 & Studen M-Werkstattpersonal & max. 10 \\ 
	\hline 
	1.9 & Lieferantenwahl Sammelbestellung & gem. Kapitel 4.5 Aufgabenstellung \\ 
	\hline 
	 
\end{tabular} 

\section{Plattform}
\begin{tabular}{|l|l|l|}
	\hline 
	Nr & Bezeichnung & Beschreibung \\ 
	\hline 
	2.1 & Gesamtlänge & 350 +/- 2cm \\ 
	\hline 
	2.1 & Drahtseil & Verzinkter Stahl, Durchmesser 3mm \\ 
	\hline 
	2.3 & Seilspannung & Via Umlenkrollen durch ein Gewicht mit einer Masse von 15kg \\ 
	\hline 
	2.4 & Startfeld & 50cm +/- 2cm, Quadratisch \\ 
	\hline 
	2.5 & Zielplatte & Gesamtmass? \\ 
	\hline 
	2.6 & Zielplatte & Wie viele konzentrische Bereiche? \\ 
	\hline 
	&  &  \\ 
	\hline 
	&  &  \\ 
	\hline 
	&  &  \\ 
	\hline 
\end{tabular} 
\section{Laufkatze}


\newpage

\printbibliography

\end{document}